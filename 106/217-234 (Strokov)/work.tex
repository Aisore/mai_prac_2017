\documentclass{mai_book}

\defaultfontfeatures{Mapping=tex-text}
\setmainfont{DejaVuSerif}
\setdefaultlanguage{russian}

%\clearpage
\setcounter{page}{216} % ВОТ ТУТ ЗАДАТЬ СТРАНИЦУ
%\setcounter{thesection}{6}% ТАК ЗАДАВАТЬ ГЛАВЫ, ПАРАГРАФЫ И ПРОЧЕЕ.
% Эти счетчики достаточно задать один раз, обновляются дальше сами


% \newtop{ЗАГОЛОВОК}  юзать чтобы вручную поменть заголовок вверху страници

\begin{document}
%\chapter{Алгоритмика и \newline программирование на \newline языке Ада}
%\newtop{П-6 Об оптимальных алгоритмах вычисления НОД}
\noindent\textbf{Доказательство}.
\newline\\
\hspace*{15pt}
\begin{tabular}{|p{12.5cm}}
По условию $b \leqslant 10^{p} - 1$ и $log_{10}(b+1) \leqslant p$. Следовательно,
$$\frac{log_{10}\sqrt{5}(b+1)}{log_{10}\phi}=\frac{log_{10}(b+1)}{log_{10}\phi}+\frac{log_{10}\sqrt{5}}{log_{10}\phi}\leqslant\frac{1}{log_{10}\phi}p+\frac{log_{10}\sqrt{5}}{log_{10}\phi}=\alpha p+\beta,$$

так как $\alpha\approx 4,7879$ меньше 5 и $\beta\approx 1,6722$ несколько меньше 2.

Итак, можно записать:$\alpha p+\beta=5p+\beta-(5-\alpha)p$ формула, в которой число $\beta - (5 - \alpha)p$ мажорируется 1, когда $p \geqslant 4$, что дает
$$\frac{log_{10}\sqrt{5}(b+1)}{log_{10}\varnothing}\leqslant\alpha p+\beta\leqslant 5p+1.$$
Чтобы завершить доказательство, остается рассмотреть случаи, когда $p=1,2$ или $3$. Простые вычисления показывают, что когда $p\in[1,3]$, имеем $\lfloor\alpha p+\beta\rfloor=5p+1$, что доказывает лемму.
\end{tabular}
%\end{proof}
\\\\
\textbf{Доказательство} (Теоремы Ламе).
\\\\
\hspace*{15pt}
\begin{tabular}{|p{12.5cm}}
Свойство 45 вместе с предыдущей леммой позволяет без труда доказать теорему Ламе. Действительно, из этих двух результатов выводим $n+1\leqslant 5p+1$, т.е. $n \leqslant 5p$. Кроме того, утверждение, состоящее в том, что $b$ имеет $p$ десятичных цифр в записи, означает $p=\lfloor log_{10}b\rfloor +1=\lceil log_{10}(b+1)\rceil$.
\end{tabular}
\\\\
Главный результат в этой теории — сложность алгоритма Евклида для целых числе логарифмическая по отношению к наименьшему из двух чисел. (Обозначение $O(log b)$ скрывает весьма существенную константу).

\begin{center}
\begin{tabular}{p{12cm}}
\textbf{Замечание.} В оценке Ламе коэффициент 5 оптимален, так как для следующих пар $(13,8)$, $(144,89)$ и $(1597,987)$, соответствующих парам $(F_{7},F_{6})$, $(F_{12},F_{11})$ и $(F_{17},F_{16})$, число итераций алго­ритма Евклида, соответственно, 5, 10 и 15. Впрочем, это свойство
не идет дальше $F_{17}$ и $F_{16}$. Действительно, для пары
\\
$$(F_{22},F_{21})=(17711,10946)$$
\\
число итераций равно 20, в то время как оценка Ламе 25. Это по­
казывает, что если коэффициент 5 оптимален, то мажорирующая
функция таковой не является.
\end{tabular}
\end{center}
\newpage
\subsection{Квазиевклидовы кольца}
\noindent Мы только что доказали, что оценка, даваемая теоремой Ламе, опти­
мальна только в некоторых особых случаях. Но даже если бы выраже­
ние, даваемое этой теоремой, было наилучшим, вопрос об оптимально­
сти алгоритма Евклида нахождения НОД двух целых чисел все равно
остался бы. Действительно, обычное евклидово деление не приводит
к оптимальному алгоритму Евклида, как видно из раздела 3.4. Цель
данного раздела — найти деления, которые оптимизируют сложность
этого алгоритма. До настоящего момента мы рассматривали, главным
образом, три класса колец, связанные следующими включениями:
$$hzhzhzhhzhz$$
Главные их характеристики, которые нас интересуют, следующие:
\begin{itemize}


\item\textit{Наличие} алгоритма Евклида и возможность эффективного вычи­
сления НОД. Прототипами евклидовых колец являются $\mathbb{Z}$ и мно­
жество многочленов над полем.

\item\textit{Наличие} НОД и разложения Безу без эффективного метода вы­
числения. Кольцо целых чисел из $\mathbb{Q}\sqrt{-19}$ является неевклидо­
вым кольцом главных идеалов, в котором, между прочим, суще­
ствует метод вычисления коэффициентов Безу [141] (кольцо це­лых $\mathbb{Q}\sqrt{-19}$, состоящее из элементов, которые являются корнями
унитарных многочленов с коэффициентами из $\mathbb{Z}$).

\item\textit{Наличие} и \textit{единственность} разложения на простые множители
(основная теорема арифметики). Кольцо $A[X]$ многочленов с ко­
эффициентами из $A$ факториально, если таковым является $A$, но
не является КГИ, если $A$ не является телом.
\end{itemize}

Хотя понятие евклидова кольца более интересное с точки зрения
эффективности, понятие алгоритма Евклида значительно сильнее. На
практике оно ограничено методами, которые были представлены выше:
\begin{itemize}


\item В КГИ известно существование НОД, но нет, вообще говоря, про­
стой эффективной процедуры его вычисления.

\item Как будет видно из следующей главы, посвященной модулям над
КГИ, что является основным содержанием главы, важным явля­
ется не вычисление НОД с помощью деления, а вычисление коэф­
фициентов Везу (что, конечно, приводит к нахождению НОД).

\item В кольце главных идеалов наличие НОД зависит не столько от то­
го, что всякий идеал главный, сколько от того, что он конечного
типа (этого достаточно).
\newpage 
\item Наконец, как показывают нижеследующие результаты, сходи­
мость некоторого метода такого, как алгоритма Евклида, не обя­
зательно связано с существованием алгоритма Евклида в рассма­
триваемом кольце.
\end{itemize}
\subsection*{(48) Определение.}
\textit{Пусть А — коммутативное унитарное кольцо.} \textbf{Квазиалгоритм} \textit{на
А есть отображение} $\varphi$ \textit {множества А х А во вполне упорядоченное мно­
жество, обладающее следующим свойством:}
\begin{center}
$\forall(a,b)\in A\times A^*, \exists(q,r)\in A\times A$ \textit{такая, что} $a = bq + r$ и $\varphi(b,r)<\varphi(a,b)$
\end{center}
\textit{Это равенство называется делением} $a$ \textit{на} $b$. \textit{Кольцо А
называется} \textbf{квазиевклидовым,} \textit{если оно допускает квазиалгоритм} $\varphi$. \textit{Говорят также, что А квазиевклидово относительно $\varphi$}
\begin{center}
\begin{tabular}{p{12cm}}
\textbf{Примечание.} Так как в случае евклидовых колец, термин \textit{алго­ритм} обозначал отображение, а не эффективный метод вычисле­
ния, то квазиалгоритм также будет полностью формальным объ­
ектом. Эта терминология нежелательна в работе, которая рас­
сматривает алгоритмы в «информатическом» смысле.
\end{tabular}
\end{center}
\subsection*{(49) Предложение.}

\textit{(i) Всякое евклидово кольцо является квазиевклидовым}.

\textit{(ii) В квазиевклидовом кольце всякий идеал конечного типа главный
(что приводит к наличию НОД двух элементов). Это означает, что
всякое квазиевклидово кольцо является кольцом Везу.}

\textit{(iii) В частности, всякое нётерово квазиевклидово кольцо без дели­
телей нуля является КГИ (в нётеровом всякий идеал конечного типа).
}
\\\\
\noindent\textbf{Доказательство.} только для (\textit{ii})
\\\\
\hspace*{15pt}\begin{tabular}{|p{12.5cm}}
Пусть идеал $А$ порожден двумя элементами. Допустим, что
$(a, b)\in A \times A$ — пара образующих идеала $I, I = Aa + Ab$, где $b$
отличен от нуля, для которой $\varphi$ имеет наименьшее значение. Мож­
но осуществить квазиевклидово деление $a$ на $b$ и записать $a=bq+r$,
где $\varphi(b,r) < \varphi(a, b)$. Это приводит, в частности, к $Aa+Ab=Ab+Ar$
и противоречит выбору пары $(a,b)$. Следовательно, такой пары с
$b\neq0$ не существует, т.е. $b=0$. Это запускает механизм индукции,
которая не вызывает особых трудностей. Наличие НОД теперь вы­
водится из того, что $Ad = Aa + Ab \Rightarrow[\delta|d \Longleftrightarrow \delta|a$ и $\delta|b]$.
\end{tabular}
\newpage
\begin{center}
\begin{tabular}{p{12cm}}
\textbf{Замечание.} Конечно, лучшее понимание квазиевклидова деле­
ния, основного для вычисления НОД, необходимо. Однако оно
не позволяет провести эти вычисления. Как показывает преды­
дущее доказательство, НОД двух элементов кольца определяется
единственным образом (это нулевой элемент пары, порождающий
сумму идеалов, для которой значение квазиалгоритма минималь­
но, что не является эффективным средством вычисления).
\end{tabular}
\end{center}

Алгоритм Евклида остается допустимым для квазиевклидовых ко­
лец и обладает той же сходимостью, что и в евклидовых кольцах.

\subsection*{(50) Предложение.}
\textit{Коммутативное унитарное кольцо А является квазиевклидовым то­
гда и только тогда, когда оно удовлетворяет следующему условию:
}
\begin{center}
 $\forall(r_{0},r_{1})\in A\times A,\exists n\in\mathbb{N},\exists(q_{1},\ldots,q_{n})\in A^n,\exists(r_{2},\ldots,r_{n+1})\in A^n$,\linebreak
\textit{такие, что:} $\forall i\in[1,n] : r_{i-1}=r_{i}q_{i}+r_{i+1}$ и $r_{n+1} = 0$.
\end{center}
\textit{Кроме того, отображение} $\varphi$ \textit{которое co всякой парой} $(r_{0},r_{1})\in A\times A$ \textit{ассоциирует наименьшее целое} $n$, \textit{для которого существует цепь псев­
доделений, оканчивающаяся нулевым остатком, является самым малым
квазиалгоритмом, определенным на А.}
\\\\
\textbf{Доказательство.}
\\\\
\hspace*{15pt}\begin{tabular}{|p{12.5cm}}
Согласно второму пункту предшествующего замечания, условие не­
обходимо. Поэтому надо показать его достаточность. Итак, пусть
$A$ — кольцо, удовлетворяющее условию предложения $50$. Ясно, что
отображение $\varphi$ есть квазиалгоритм для $A$. Действительно, пусть $a$,
$b$ и $r$ такие, что минимальная цепь псевдо делений для пары $(a,b)$
начинается с $a=bq+r$. Тогда, принимая во внимание минималь­
ность $\varphi$, имеем $\varphi(b,r)+1\leqslant\varphi(a,b)$ и, следовательно, $\varphi(b,r) < \varphi(a,b)$.
Проверка того, что указанный квазиалгоритм минимален, является простой формальностью.
\end{tabular}
\\

Итак, данное предложение позволяет построить квазиалгоритм для
квазиевклидова кольца. Это построение (если оно эффективно) опре­
деляет самый быстрый метод вычисления НОД через алгоритм «по в­
клиду» в квазиевклидовом кольце. Действительно, Лазар [112] доказал
следующие свойства:
\subsection*{(51) Теорема (Лазара)}
\textit{(i) Обычное евклидово деление в K[X] является евклидовым делени­
ем, согласно минимальному квазиалгоритму в K[X]. Алгоритм Евкли­
да, следовательно, является самым быстрым методом вычисления НОД}
\textit{двух многочленов с коэффициентами в поле через последовательные
деления.}

\textit{(ii) В $\mathbb{Z}$ всякое евклидово деление с самым малым остатком является делением согласно минимальному кваэиалгоритму в $\mathbb{Z}$.}

\textit{(iii) Точнее, в $\mathbb{Z}$ деление $a = bq+r$ есть деление по минимальному квазиалгоритму тогда и только тогда, когда $|r| < |b|/\phi$ ($\phi$ является золотым числом и $1/\phi\approx0,6180339\ldots)$.}

\textbf{Доказательство.}

\hspace*{15pt}\begin{tabular}{|p{12.5cm}}
Доказательство пункта (i) очень простое и фигурирует в упражне­
ниях 42 и 43, находящихся в конце главы. Пункты (i) и (iii) не очень
интересны для доказательства и не представляют больших трудно­
стей. Например, можно проиллюстрировать пункт (iii). Пусть для
вычисления НОД даны числа 4215 и 1177. Вот различные этапы ал­
горитма Евклида. Слева используется деление с наименьшим остат­
ком, а справа деление, для которого отношение остатка к частному
может превысить 0,5, все еще не превышая 1/$\phi$:
$$hzhzhzhz$$
В решении по этому последнему алгоритму 4 деления, отмеченные
звездочкой, дают остатки, абсолютное значение которых больше
половины делителя, оставаясь в границах теоремы Лазара, а число
итераций остается равным 9 (значение минимального квазиалго­
ритма для этих двух целых чисел).
\end{tabular}
\subsection{Вычисление НОД нескольких целых чисел: теорема Дирихле}
Когда необходимо вычислить НОД нескольких чисел, а не только двух,
можно применить несколько методов:
\newpage
• Распространение алгоритма Евклида, базирующегося на следую­
щих свойствах:
\textit{(i)} НОД$(0,\ldots, 0, a, 0 ,\ldots, 0) = a$,
\textit{(ii)} НОД$(u_{1},\ldots,u_{i},\ldots,u_{n})$ = НОД$(u_{1}$ mod $u_{i},\ldots,u_{i},\ldots,u_{n}$ mod $u_{i}$) при $u_{i}\neq0$.
За подробностями этого метода читатель может обратиться к
упр. 24.

• Следующий метод заключается в повторном применении алгорит­
ма Евклида для двух целых чисел. Он основывается на следую­
щем свойстве: НОД($u_{1},\ldots,u_{n}$) = НОД($u_{1}$,НОД($u_{2},\ldots,u_{n}$)), ко­
торое порождает рекурсивный алгоритм вычисления НОД. Имен­
но, НОД($u_{1},\ldots,u_{n}$) = НОД(НОД($u_{1},\ldots,u_{n}),u_{3},\ldots,\u_{n}))$, что являет­
ся основой соответствующего итеративного алгоритма.

Этот последний метод не только упрощает реализацию вычисле­
ния — действительно, достаточно несколько раз применить уже реа­
лизованный алгоритм — но и имеет неоспоримое достоинство с точки
зрения эффективности. Неформально говоря, главное заключается в
следующем: выбирают два числа в последовательности, для которой
надо вычислить НОД. Затем, сделав первое вычисление, заменяют два
выбранные числа на их НОД и повторяют алгоритм. Выигрыш в эф­
фективности получается из того, что как только находят НОД, рав­
ный единице, вычисление может быть прервано. Неиспользованные чи­
сла ничего не могут добавить к полученным результатам. К тому же
это явление довольно распространенное, потому что, как утверждает
теория Дирихле, более, чем в 60% случаев два целых числа, взятые слу­
чайно, взаимно просты. Продолжение этого раздела посвящается двум
доказательствам теоремы Дирихле. Одно — эвристическое (короткое
и неправильное), а другое — более длинное, но верное. Второе доказа­
тельство требует введения функции Мёбиуса, и мы воспользуемся слу­
чаем доказать формулу обращения Мёбиуса — ту формулу, которая
уже была использована в разделе 4.2.

\textbf{(52) Теорема (Дирихле)}

\textit{Если $u$ и $v$ — два натуральных числа, выбранные случайно, то веро­
ятность того, что они взаимно просты, равна $6/sqrt{\pi}\approx 0,607927$. Более формально, если}

$H_{1}^{n}=\{(u,v)\in\mathbb{N}^{*2}/1\leqslant u\leqslant n,1\leqslant v\leqslant n$ и НОД$(u,v)=1\}$,

\textit{то $p=\lim\limits_{n\to\infty}\frac{\#H_{1}^{n}}{1}=\frac{6}{sqrt{n}}$, где $\#H_{1}^{n}$ означает мощность множества $H_{1}^{n}$}
\newpage
Обозначения, использующиеся в следующих ниже доказательствах,
таковы. Для натурального числа $d$ и вещественного положительного $x$
определим:

$H_{d}=\{(u,v)\in\mathbb{N}^{*2}/$ НОД$(u,v)=d\}$,

$H_{d}^x=\{(u,v)\in\mathbb{N}^{*2}/1\leqslant u\leqslant x,1\leqslant v, \leqslant x$ и НОД$(u,v)=d\}$

В этих обозначениях $x$ часто будет заменяться на натуральное число,
как в формулировке теоремы Дирихле. Множество $H_{1}$ необходимо для
оценки функции распределения в $\mathbb{N}^{*2}$.

\textbf{Эвристическое доказательство} (теоремы Дирихле).

Утверждение НОД$(u,v) = d$ равносильно тому, что $u$ и $v$ кратны
$d$ и что НОД$(u/d,v/d) = 1$ (т.е. $(u/d,v/d)\in H_{1})$. К тому же для
всякого натурального $n$ множества $H_{d}^{nd}$ и  $H_{1}^n$ имеют одну и ту же
мощность. Следовательно, для $х > 0$:
$$\frac{\#H_{d}^x}{x^2}=\frac{\#H_{1}^{x/d}}{x^2}=\frac{\#H_{1}^{x/d}}{(x/d)^2}=\frac{\#H_{1}^{x/d}}{(x/d)^2}\times\frac{1}{d^2}$$
Переходя к пределу, получим:
\begin{center}
$\lim\limits_{x\to\infty}\frac{\#H_{d}^x}{x^2}=\frac{1}{d^2}\times\left(\lim\limits_{x\to\infty}\frac{\#H_{1}^x}{x^2} \right) = \frac{p}{d^2}$,
\end{center}
где последнее число — «вероятность» того, что два числа имеют
наибольший общий делитель $d$. С другой стороны, можно запи­
сать $\mathbb{N}^{*2}=\cup_{d=1}^{\infty}H_{d}$, где
 объединение является объединением не-
пересекающихся множеств ($H_{d}$ — есть множество пар целых на­
туральных чисел с НОД равным $d$). Тогда можно заключить, что\linebreak
$1 =\Sigma_{d=1}^{\infty} p/d^2 = p\Sigma_{d=1}^{\infty}1/d^2=p\pi^2/6$.

Настоятельно просим читателя найти ошибку в этом интуитивном
доказательстве.

А сейчас переходим к доказательству теоремы Дирихле. Для этого
доказательства нужна функция Мёбиуса вне связи с теорией арифме­
тических функций, где она обычно появляется.

\textbf{(53) Свойство.}

\textit{Назовем функцией Мёбиуса функцию $\mu$, определенную на $\mathbb{N}^{*}$ следу­
ющим образом:
}

$$hzhzhzhzhhzhzhzhzhzhz$$
\newpage
\textit{Эта функция тесно связана с частичным упорядочиванием на множе­
стве целых чисел с помощью отношения делимости. Для целого числа
$d>1$ функция $\mu$ удовлетворяет соотношению $\Sigma_{k|d}\mu(k)=0$.}

\textbf{Доказательство.}

Пусть $d=p_{1}^{\alpha_{1}}\ldots p_{r}^{\alpha_{r}}$
 — каноническое разложение $d$. Достаточно
рассмотреть только делители $d$, свободные от квадратов, так как
функция $\mu$ на других делителях исчезает, т.е. такие делители числа
$d$, которые разлагаются в произведение простых $p_{i}$ с показателями
0 или 1. Следовательно, чтобы найти такие делители, достаточно
выбрать $j$ различных чисел $р_{i}$ и получить
\begin{center}
$\sum\limits_{k|d}\mu(k)=\sum\limits_{j=0}^r(-1)^j\times (hzhzhz) =(1+(-1))^r$,
\end{center}
согласно определению $\mu$.

Теперь мы получим формулу обращения Мёбиуса в не совсем при­
вычной форме

\textbf{(54) Предложение.}

\textit{Пусть $f$ и $g$ — две функции от $\textit{R}_{+}^{•}$ со значениями в некоторой ад­
дитивной абелевой группе. Тогда
}
\begin{center}
$f(x)=\sum\limits_{k=1}^{|x|} g\left(\frac{x}{k}\right) \Longleftrightarrow g(x)=\sum\limits_{k=1}^{|x|}\mu(k)\times f\left(\frac{x}{k}\right)$.
\end{center}
\textbf{Доказательство.}
Используемый метод заключается в перенесении выражения от $f$,
задаваемого первой формулой, во вторую:
\begin{center}
$\sum\limits_{k=1}^{|x|}\mu(k)\times f\left(\frac{x}{k}\right) =\sum\limits_{k=1}^{|x|}\left(\mu(k)\times\sum\limits_{k=1}^{\rfloor x/k \lfloor} g\left(\frac{x/k}{p}\right)\right)=\sum\limits_{hzhz}\mu(k)\times f\left(\frac{x}{kp}\right)$.
\end{center}
Учитывая, что = $\lfloor\lfloor x \rfloor/k\rfloor = \lfloor x/k\rfloor$, можно сделать следующую замену
переменных: $h=kp$ c $1\leqslant h\leqslant x$ и $k|h$ , что дает
\newpage
\begin{center}
$\sum\limits_{k=1}^{\lfloor x \rfloor}\times f\left(\frac{x}{k}\right)=\sum_{h=1}^{\lfloor x\rfloor}\left( g\left(\frac{x}{h}\right)\times\sum\limits_{k|h}\mu(k)\right)$.
\end{center}
Согласно свойству 53, в этой сумме есть только одно ненулевое сла­
гаемое, именно то, которое отвечает значению $h=1$ и, следователь-
но $\sum\nolimits_{k=1}^{\lfloor x\rfloor}\mu(k)\times f(x/k)=g(x)$.

Доказательство теоремы Дирихле будем осуществлять в три этапа.

\textbf{(55) Лемма.}

\textit{Пусть $x$ — вещественное положительное число. Обозначим через $q_{x}$
число элементов множества $H_{1}^x$. Тогда $q_{x}=\sum\nolimits_{k\geqslant 1}\mu(k)\times\lfloor x/k\rfloor^2$, фор­
мула, в которой, на первый взгляд, бесконечное число слагаемых, но
только конечное их число отлично от нуля.
}

\textbf{Доказательство}

Пусть $q_{d,x}$ — множество элементов множества $H_{d}^x$.Множество пар
натуральных чисел, не превосходящих $x$, есть, как это было видно
в эвристическом доказательстве, теоретико-множественная сумма
непересекающих подмножеств $H_{d}^x$, где $d$ изменяется от 1 до $\lfloor x\rfloor$: $\lfloor x\rfloor^2=\sum\nolimits_{d=q}^{\lfloor x\rfloor}q_{d,x}=\sum\nolimits_{d=1}^{\lfloor x\rfloor}q_{\lfloor x/d\rfloor}$, так как $\#H_{d}^x=\#H_{1}^{\lfloor x/d\rfloor}$. При­
меним к этому выражению формулу обращения, фигурирующую в
предложении 54, отождествляя функцию $f$ с функцией $x\mapsto\lfloor x\rfloor^2$
 и функцию $g$ с функцией $x\mapsto q_{x}$, и получим искомый результат.
 
Следующий этап доказательство требует (это было неизбежно) вычисления пределов и суммы ряда.

\textbf{(56) Лемма.}

\textit{В предшествующих обозначениях для натурального числа $n$ имеем}
\begin{center}
$\lim\limits_{n\to\infty}\frac{q_{n}}{n^2}=\sum\limits_{k=1}^{\infty}\mu(k)/k^2$.
\end{center}
\textbf{Доказательство.}

Ряд, фигурирующий в правой части формулы, является, очевидно,
абсолютно сходящимся, так как функция Мёбиуса мажорируется по
абсолютной величине единицей. Итак, достаточно оценить разность
между общими членами этих двух последовательностей и показать,
что она стремится к нулю. Имеем:
\newpage
\begin{center}
$\sum\limits_{k=1}^{n}\frac{\mu(k)}{k_{2}}-\frac{q_{n}}{n^2}=\sum\limits_{k=1}^n\mu(k)\times \left(\frac{1}{k^2}-\left\lfloor\frac{n}{k}\right\rfloor^2\times\frac{1}{n^2}\right)$.
\end{center}
Кроме того, для всякого вещественного положительного числа $x$
имеем $0\leqslant x-\lfloor x\rfloor <1$ и, следовательно, $0\leqslant 1/k-\lfloor n/k\rfloor/n\leqslant 1/n$. В
этих условиях
\begin{center}
$0\leqslant\frac{1}{k^2}-\frac{1}{n^2}\times\lfloor\frac{n}{k}\rfloor^2=\left(\frac{1}{k}-\frac{1}{n}\times\lfloor\frac{n}{k}\rfloor\right)\times\left(\frac{1}{k}+\frac{1}{n}\times\left[\frac{n}{k}\right]\right)\leqslant\frac{1}{n}\times\frac{2}{k}$.
\end{center}
Перенося эти значения в разность для мажорирования, получаем:
\begin{center}
$\left(\frac{q_{n}}{n^2}-\sum\limits_{k=1}^{n}\frac{\mu(k)}{k^2}\right)\leqslant\frac{2}{n}\times\sum\limits_{k=1}^{n}\frac{1}{k}\leqslant\frac{2\log_n}{n}$
\end{center}
величина, стремящаяся к нулю, когда $n$ стремится к бесконечно­
сти. Итак, последовательность $(q_{n}/n^2)_{n\in\mathbb{N•}}$. сходится и имеет тот же предел, что и ряд.

Для того, чтобы закончить доказательство теоремы Дирихле, оста­
ется вычислить сумму ряда с общим членом $\mu(k)/k^2$. Для этого дока­
жем, что $(\sum\nolimits_{k=1}^{\infty}\mu(k)/k^2\times(\sum\nolimits_{k=1}^{\infty}1/k^2)=1$.
Оба рассматриваемых ряда — абсолютно сходящиеся, и потому можно изменить порядок суммирования следующим образом:
\begin{center}
$\left(\sum\limits_{k=1}^{\infty}\frac{\mu(k)}{k^2}\right)\times\left(\sum\limits_{m=1}^{\infty}\frac{1}{m^2}\right)=\sum\limits_{k=1}^{\infty}\sum\limits_{m=1}^{k=1}\frac{\mu(k)}{m^2 k^2}=\sum\limits_{d=1}^{\infty}\left(\sum\limits_{k|d}\mu(k)\right)\times\frac{1}{d^2}$.
\end{center}
По уже доказанному свойству 53 самая внутренняя сумма нулевая, за
исключением случая, когда $d=1$. Сумма ряда с общим членом $1/k^2$ 2
равна $\pi^2/6$. Теорема Дирихле доказана.

Конец этого раздела посвящен классической формуле обращения
Мёбиуса.

Рассмотрим множество $\mathbb{N}^*$, упорядоченное с помощью отношения
делимости. В этой структуре 1 — наименьший элемент и всякий ин­
тервал $[a,b]$ конечен. ($\mathbb{N}^*, |)$ — упорядоченное множество, являющееся
локально конечным.

Поэтому на множестве функций $\digamma$, определенных на $\mathbb{N}^*$ со значени­
ями в $A$, можно ввести внутренний закон композиции.
\newpage
\subsection*{(57) Определение}

\textit{На $\digamma$ определен закон внутренней композиции *, называемый арифметическим произведением, по следующему правилу:}
$$\forall f\in\digamma,\forall g\in\digamma,\forall n\in\mathbb{N}^{*},\;\;\;(f*g)(n)=\sum\limits_{d|n}f(d)g\left(\frac{n}{d}\right).$$
\subsection*{(58) Свойства (произведения *)}

\textit{(i) Произведение ассоциативно и коммутативно.\\
(ii) Функция Кронекера $\delta$, определяемая как $\delta(1)=1$ и $\delta(n)=0$, если
$n>1$, есть нейтральный элемент для произведения $*$.\\
(iii) Элемент $f\in\digamma$ обратим для операции $*$ тогда и только тогда,
когда $f(1)$ обратим.\\
(iv) Множество $\digamma$, наделенное обычным сложением функций и опера­
цией $*$, является коммутативным унитарным кольцом.}
\\

\noindent\textbf{Доказательство} (только для пункта (\textit{iii}).
\\
\hspace*{15pt}
\begin{tabular}{|p{12.5cm}}
Пусть $f\in\digamma$ такой, что $f(1)\in U(A)$. Определим $g$ по индукции следующим образом:
$$g(1)=f(1)^{-1}\;\;\;\text{и}\;\;\;g(n)=-f(1)^{-1}\sum\limits_{hzhz}g(d)f\left(\frac{n}{d}\right)\;\;\;\text{при}\;n>1$$
Простая проверка показывает, что $g$ — обратный элемент для $f$.
Остальные утверждения теоремы немедленно выводятся из опреде­ления арифметического произведения.
\end{tabular}
\subsection*{(59) Свойство.}

\textit{Пусть $\xi$ — элемент из $\digamma$, определяемый по правилу $\xi(n)=l,\forall n\in\mathbb{N}^*$. Тогда в $(\digamma,*)$ функции $\xi$ и$\mu$ обратны друг другу.}

Проблема, решаемая с помощью формулы обращения Мёбиуса, следующая. Пусть $g$ — функция из $\mathbb{N}^*$ в $A$ и пусть $f$ — функция, определенная по правилу $\sum\nolimits_{d|n}g(g)$ для $n\in\mathbb{N}^*$. Можно ли в этом случае выразить функцию $g$ через $f$? Другими словами, можно ли найти обращения этой формулы? Ответ дает следующая

\subsection*{(60) Теорема (Формула обращения Мёбиуса).}

\textit{Пусть $f$ и $g$ — две функции в $\digamma$ , такие, что для любого $n\in\mathbb{N}^*$ справедливо соотношение $f(n)\;=\;\sum\nolimits_{d|n}g(d)$. Тогда можно выразить $g$
через функцию $f: g(n)=\sum\nolimits_{d|n}\mu(d)f(n/d)$, где $\mu$ — функция Мёбиуса.}
\newpage
\textbf{Доказательство} (формулы обращения Мёбиуса).

Выражение $f$ через функцию $g$ описывается в терминах умножения
$*: f = \xi*g$, а так как $\xi$ и $\mu$ взаимно простые элементы относительно
 операции * (свойство 59), то получаем $g= \mu*f$.
 
Более общие сведения по теории арифметических функций можно
получить, обратившись к монографии Бержа [19].
\section{Расширенный алгоритм Евклида}
Этот раздел посвящен изучению эффективного метода получения ко­
эффициентов Безу в евклидовом и квазиевклидовом кольце. Надо от­
метить, что квазиевклидовый случай — не единственная возможность
для построения таких алгоритмов (известным примером является не­
евклидово кольцо главных идеалов кольца целых алгебраических чисел
в $\mathbb{Q}(\sqrt{-19})$). В следующей главе мы увидим, что наличие коэффициен­
тов Безу является главным ключом к классификации модулей над коль­
цом главных идеалов (теория инвариантных множителей): кто умеет их
вычислять, тот умеет решать эффективным образом задачи линейной
алгебры над кольцом главных идеалов.
\subsection{Вычисление коэффициентов Безу
в квазиевклидовом кольце}
Для квазиевклидова кольца $A$ разновидность алгоритма Евклида, пред­ложенная в разделе 3.2, позволяет вычислить коэффициенты Безу. Ис­пользуемые обозначения те же, что в разделе 3.2. Алгоритм, приме­
ненный к паре чисел $a,b$, порождает последовательность $r_{0\leqslant i\leqslant n+1}$ такую, что
$$r_{i-1}=r_{i}q_{i}+r_{i+1}\;\;\text{для}\;1\leqslant i\leqslant n,\;\;\text{где}\;r_{0}=a,r_{1}=b,r_{n+1}=0.$$
Элемент $r_{i+1}$ является линейной комбинацией $r_{i}$ и $r_{i-1} (r_{i+1}\in Ar_{i}+Ar_{i-1})$. Так как $r_{0}=1\cdot a+0\cdot b$, $r_{1}=0\cdot a+1\cdot b$, то по предыдущему
рекуррентному соотношению для г< получаем, что $r_{n} = \text{НОД}(a,b)$ — линейная комбинация $a$ и $b$. Точнее, предполагая, что $r_{i}=u_{i}a+v_{i}b$, получаем:
$$r_{i+1}=r_{i-1}-q_{i}r_{i}=(u_{i-1}-q_{i}u_{i})a+(v_{i-1}-q_{i}v_{i})b.$$
\newpage
Из этих формул легко получается рекуррентная последовательность:
$$hzhzhzhhzh$$
из которой теперь и следует классический результат: $r_{n} = \text{НОД}(a, b) =
u_{n}a+v_{n}b$. Эти соотношения приводят, к тому же, к рекуррентному
алгоритму, изображенному ниже, в котором тройка $(u,v,r)$ соответ­
ствует $(u_{i}, v_{i}, r_{i})$ и тройка $(u', v', r')$ соответствует $u_{i+1},v_{i+1},r_{i+1}$. Переменная $i$, бесполезная для алгоритма, присутствует в комментариях
только для того, чтобы придать смысл утверждениям.
$$ada\;ada\;ada$$
Вот другое доказательство, основанное на эквивалентном предста­влении того же алгоритма. Для этого все рекуррентные соотноше­ния (6) запишем в матричной форме:
$$matrix\;matrix$$
Эти равенства дают:
$$matrix\;matrix$$
Затем:
$$matrix\;matrix$$
\newpage
$$matrix\;ada$$

Следовательно, для $i=n$ имеем: $a=(-1)^nv_{n+1}r_{n},\;b=(-1)^{n+1}u_{n+1}r_{n}$ и $r_{n}=u_{n}a+v_{n}b$. Последние соотношения \textit{явно} показывают, что $r_{n}$ является, с одной стороны, общим делителем, а с другой — линейной
комбинацией $a$ и $b$. Это порождает новое \textit{эффективное} доказательство,
поскольку $r_{n}$ является НОД $a$ и $b$.
Этот подход позволяет построить самодостаточный алгоритм (3),
в котором больше не фигурирует переменная $i$.
$$tabletable$$
В таблице 3 приведен пример вычисления коэффициентов Безу в $\mathbb{Z}$. Этот пример может быть полезен для понимания следующего парагра­фа. В рассматриваемом примере $a$ = 1292, $b$ = 798, их НОД = 38 и найденные коэффициенты Безу $u$ = — 8 и $v$ = 13 (подчеркнутые числа).
\newpage
Коэффициенты Безу часто применяются для вычисления обратного элемента в $\mathbb{Z} /n\mathbb{Z}$. Пусть, например, требуется обратить класс 34 в $\mathbb{Z}/235\mathbb{Z}$ (34 взаимно просто с 235, следовательно, обратимо по моду­
лю 235): алгоритм Безу дает соотношение $1 = 11\times235 — 76\times34$, и
обратным к 34 по модулю 235, следовательно, является —76 = 159. Опе­рация обращения часто необходима в модулярной арифметике. Иногда эта конструкция требуется и в других кольцах, например, в кольце целых чисел Гаусса. Так, алгоритм Безу, примененный в $\mathbb{Z}[i]$ к числам $23 + 14i$ и $7 + 5i$, дает $1=(—3 + 2i)\times(23+14i)+(9—7i)\times(7+5i)$ и, сле­довательно, обратным к элементу $7 + 5i$ по модулю $23+14i$ будет $9—7i$.
\subsection{Мажорирование коэффициентов Безу в $\mathbb{Z}$}
Равенства $ua+vb=(u-kb)a+(v+ka)b$ и для $d$, делящего $a$ и $b$, $ua+vb=(u-kb/d)a+(v+ka/d)b$ показывают, что существует много пар $(u,v)$, для которых НОД$(a,b)=ua+vb$. Расширенный алгоритм Евклида, полученный в предыдущем разделе, позволяет вычислить такую пару $(u,v)$, что, \textit{за исключением лишь некоторых особых случаев,}выполняются неравенства $|u|\leqslant|b/2d|$ и $|v|\leqslant|a/2d|$.

Эти оценки являются объектом исследования для следующего предложения. Единственность такой пары $(u,v)$, удовлетворяющей указанным неравенствам (упр. 33), доказывает, что пара Безу, получаемая
алгоритмом Евклида, является самой «красивой».
\subsection*{(61) Предложение.}

\textit{(i) Пусть $a$ и $b$ — различные строго положительные целые числа, и пусть $d$ их НОД. Пусть $(u_{i})_{0\leqslant i\leqslant n+1}$ — последовательно­сти, полученные расширенным алгоритмом Евклида. В этих условиях последовательности $|u_{i}|_{1\leqslant i\leqslant n+1}$ и $|v_{i}|_{0\leqslant i\leqslant n+1}$ являются возрастающими, не переполняют разрядную сетку машины — в предположении, что $a$ и $b$ представимы машинными кодами — и указанный алгоритм дает коэффициенты Безу $u$, $v$, удовлетворяющие оценкам:}
$$|u|\leqslant|b/2d|,\;\;\;|v|\leqslant|a/2d|.$$

\textit{(ii) Если $(a, b)$ — пара целых чисел, отличная от $(0,a)$, $(a, 0)$ и $(a,\pm а)$, то существуют коэффициенты Безу, удовлетворяющие неравен­ствам (7)}
\newpage
\textbf{Доказательство.}

Напомним классические обозначения:
$$r_{i-1}=r_{i}q_{i}+r_{i+1},\;\;u_{i+1}=u_{i-1}-u_{i}-u_{i}q_{i},\;\;v_{i+1}=v_{i-1}-v_{i}q_{i},\\\text{и}\;\;u_{0}=v_{1}=1,\;\;u_{1}=v_{0}=0.$$
Легко убедиться, что $u_{2i}\geqslant0$ и $u_{2i+1}\leqslant0$; значит, $|u_{i+1}\geqslant|u_{i}q_{i}|$, откуда видно (ввиду $q_{i}>0$), что последовательность $(|u_{i}|)$ возрастающая для $i\geqslant1$. В конце алгоритма имеем $|u_{n+1}|\geqslant|q_{n}u_{n}|$, где $q_{n}\geqslant2$, что неверно только для случаев, выписанных в явном виде в (ii). Однако $|u_{n+1}|=a/d$, что заканчивает доказательство (для $v_{i}$ доказательство аналогично.

\section{Факториальность кольца многочленов}
Прежде чем закончить эту главу, «пробежимся» по кольцам многочленов, что позволит построить приемлемые алгоритмы вычисления НОД с эффективными оценками трудоемкости. Но сначала немного теории.
\subsection*{(62) Теорема.}

\textit{(i) Если $A$ — унитарное нётерово коммутативное кольцо, то кольцо
многочленов $A[X]$ нётерово. То же верно и для кольца $A[X_{1},\ldots,X_{n}]$.\\
(ii) Если $K$ — поле, то $K[X_{1},\ldots,X_{n}]$ нётерово.\\
(iii) Кольцо многочленов $\mathbb{Z}[X_{1},\ldots,X_{n}]$ с целыми коэффициентами нётерово.}
Для доказательства теоремы нам понадобятся некоторые простые результаты. Пусть $n$ — натуральное число, $I$ — идеал в $A[X]$. Обозна­чим через $dom_{n}(I)$ часть $A$, состоящую из коэффициентов при старших (доминирующих) членах многочленов из $I$, имеющих степень в точности равную $n$, и к которой добавлена константа $0$.
\subsection*{(63) Лемма.}

\textit{(i) $dom_{n}(I)$ — идеал в $A$.\\
(ii) Если $n\leqslant m$, то $dom_{n}(I)\subset dom_{m}(I)$(рассмотреть $X^{m-n}P$ для $Р$ в $I$, имеющего степень $n$).\\
(iii) Если $I\subset J$, то $dom_{n}(I)\subset som_{n}(J)$.}
\subsection*{(64) Лемма.}
\textit{Пусть $I\subset J$ — два идеала в $A[X]$, такие, что для всякого $n$: $dom_{n}(I) = dom_{n}(J)$. Тогда $I=J$.}
\newpage
\textbf{Доказательство.}

Пусть $P$ — элемент $J$. Если $P$ степени $0$, то очевидно (так как $dom_{0}(I)=dom_{0}(J)$), что $P\in I$. В остальных случаях будем исполь­зовать индукцию по степени $n$ полинома $P$. Пусть $P=aX^n+\ldots$ и $a\in dom_{n}(J)=dom_{n}(I)$. Поэтому существует $Q\in I$, такой, что $Q=aX^n+\ldots$. Многочлен $P — Q$ имеет степень, меньшую, чем $n$, и принадлежит $J$. По предположению индукции получаем $Р — Q\in I$, а тогда $P\in I$.

\textbf{Доказательство теоремы 62}

Рассмотрим возрастающую последовательность $(I_{i})$ идеалов в $A[X]$. Семейство идеалов $(dom_{n}(I_{i}))_{i,n}$ имеет максимальный элемент (но не максимум на данный момент) $dom_{n_{0}}(I_{i_{0}})$. Следовательно, для всякого $n\geqslant n_{0}$ и для всякого $i\geqslant i_{0}\:dom_{n}(I_{i})=dom_{n_{0}}(I_{i_{0}})$. Рассмотрим таблицу:
$$tabletable$$
Существование $i_{0}$ и $n_{0}$ означает, что столбцы таблицы, ранг которых превышает $n_{0}$, стабилизируются, начиная с линии $i_{0}$. Более того, стабилизируется всякий столбец с номером $n<n_{0}$. Поэтому, строки предыдущей таблицы, начиная с некоторого индекса $q$, совпадают: для всякого $i\geqslant q$ имеем $dom_{n}(I_{i})=dom_{n}(I_{q})$, и по лемме 64 это доказыва­ет, что $I_{i}=I_{q}$. Последовательность идеалов в $A[X]$ стабилизируется. Следовательно, $A[X]$ нётерово.

Перейдем теперь к свойствам разложения.

\subsection*{(65) Лемма \text{(гаусса)}}

\textit{Пусть простой элемент $p$ кольца $A$ делит произведение многочленов
$P$ и $Q$ над $A$. Тогда $p$ делит $P$ или $Q$.}

\textbf{Доказательство.}

Пусть $p$ не делит ни $P$, ни $Q$. Обозначим через $a_{i}$ и $b_{j}$ такие коэф­фициенты $P$ и $Q$, соответственно, что $i$ и $j$ — наименьшие номера.
\newpage
для которых $р\nmid a_{i}$ и $р\nmid b_{j}$. Тогда $р\nmid\sum\nolimits_{k+l=i+j}a_{k}b_{l}$, так как $p$ делит все слагаемые этой суммы, кроме первого. Противоречие. В действительности лемма Гаусса не утверждает ничего, кроме того, что «если $А/(р)$ без делителей нуля, то это же утверждение верно и для $A[X]/(p)$».
\subsection*{(66) Определение.}

\textit{Пусть $A$ — факториальное кольцо. Назовем \textbf{содержанием} многочлена $P$ с коэффициентами из $A$ \text{НОД} его коэффициентов и обозначим его через $c(P)$. Многочлен $P$ называется \textbf{примитивным}, если его коэффициенты взаимно просты, т.е. если его содержание равно 1. \textbf{Примитивная часть} $P$ равна $P/c(P)$, это примитивный множитель.}
\subsection*{(68) Следствие.}
\textit{Пусть Р — \textbf{примитивный} многочлен с коэффициентами в факториальном кольце $A, Q$ — другой многочлен. Если $P$ делит $Q$ над полем частных кольца $A$, то он делит $Q$ и в $A[X]$. В частности, если для $a\in A^*\;\;P$ делит $aQ$, то $P$ делит $Q$}
\subsection*{(69) Теорема (Гаусса)}
\textit{Пусть $A$ — факториальное кольцо, $K$ — его поле частных, $S_{a}$ — система представителей неприводимых элементов из $A$ (т.е. такая си­стема, по которой можно единственным образом разложить любой эле­мент из $A$). Пусть $S'$ — система представителей неприводимых многочленов в $K[X]$ с коэффициентами из $A$, являющихся примитивными. Тогда $S'\cup S_{A}$ — система представителей неприводимых элементов в
$A[X]$. В частности, $A[X]$ факториально.}
\newpage
\textbf{Доказательство.}

Пусть $P\in A[X]$ — многочлен положительной степени (в противном случае $P$ раскладывается по системе $S_{a}$). В $K[X]$, являющемся кольцом главных идеалов, а следовательно, факториальным, многочлен $P$ можно разложить в произведение неприводимых:
$$P=a\prod\limits_{Q\in S'}Q^{\alpha_{Q}},\;\;\text{где}\;\;a\in K^*\;\text{и}\;\alpha_{Q}\in\mathbb{N},$$
$a$ можно записать в виде $a=p/q$, где $p,q\in A$. Следовательно, $qP=p\prod Q^{\alpha_{Q}}$, что является равенством в $A[X]$. Согласно следствию 68 $\prod Q^{\alpha_{Q}}$являющийся примитивным многочленом, делит $qP$, а потому делит $P$ и $a\in A$ (в силу единственности разложенияв $K[X]$). Но тогда получим разложение в $A[X]$!

\noindent Разложение в $K[X]$ единственно, так как $K[X]$ факториально. Следовательно, разложение в $A[X]$ единственно.
\subsection*{(70) Следствие.}

\textit{(i) Если $A$ — факториальное кольцо, то это же верно и для $A[X_{1},\ldots,X_{n}$ и результат, разумеется, верен, если $А$ — поле.\\
(ii) В частности, $\mathbb{Z}[X_{1},\ldots,X_{n}]$ — факториальное кольцо.}

\textbf{Примечания.}

1. Теорема Гаусса позволяет охарактеризовать неприводимые элементы в $A[X_{1},\ldots,X_{n}]$. А именно:

• неприводимые константы в $A$,

• неприводимые примитивные многочлены над полем дробей кольца $A$.
2. Всякое кольцо многочленов над факториальным кольцом факториально. Однако кольцо многочленов над КГИ не является, вообще говоря, КГИ. ($A[X]$ — кольцо главных идеалов тогда и только тогда, когда $A$ — поле).

Идеал $I=(X+2)\mathbb{Z}[X]+X\mathbb{Z}[X]$ в кольце $\mathbb{Z}[X]$ не является главным, хотя он и максимален, так как это ядро сюръективного морфизма $x$ из $\mathbb{Z}[X]$ на $\mathbb{Z}/2\mathbb{Z}$, который каждому многочлену ставит в соответствие класс четности его постоянного коэффициента. Итак, $\mathbb{Z}[X]/I$ тело, и $I$ — максимальный.

Мы закончим эту, немного абстрактную, часть критерием неприводимости многочлена в факториальном кольце: критерий Эйзенштейна.
\end{document}