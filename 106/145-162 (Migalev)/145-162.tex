\documentclass{../../template/mai_book}

\defaultfontfeatures{Mapping=tex-text}
\setmainfont{DejaVuSerif}
\setdefaultlanguage{russian}

% === move to mai_book.cls===
\usepackage{mathtools}
\DeclarePairedDelimiter{\abs}{\lvert}{\rvert}
\DeclarePairedDelimiter{\floor}{\lfloor}{\rfloor}
% ===========================

\begin{document}

\lhead{\small\textit{Решения упражнений}}
\rhead{145}

\noindent
[1, $n$] в [1, $n$], при этом коэффициент $\mu_\alpha$ определяется суммой

\begin{equation*}
\mu_{\alpha} = \sum_{\epsilon_i\in\{0,1\}} (-1)^{\epsilon_1+...+\epsilon_n} \epsilon_{\alpha(1)}...\epsilon_{\alpha(n)}.
\end{equation*}

\noindent
Если $\alpha$ — перестановка, коэффициент $\mu_\alpha$ равен $(—1)^n$, поскольку единственный ненулевой член суммы — это тот, для которого все $\epsilon_i$, равны 1. В противном случае можно разложить $\mu_\alpha$ следующим образом:

\begin{equation*}
\sum_{\epsilon_i, i \neq k} (-1)^{\epsilon_1+...+\epsilon_n} \sum_{\epsilon_k=0,1} (-1)^{\epsilon_k} \epsilon_{\alpha(1)}...\epsilon_{\alpha(n)},
\end{equation*}

\noindent
где $k$ — элемент из [1, $n$], который не принадлежит образу $\alpha$, и хорошо видно, что внутренний член (сумма по $\epsilon_k$) равен нулю; это доказывает, что $\mu_\alpha$ в этом случае равно нулю. Второе искомое выражение (суммы которого записываются на множествах) просто выводится из первого, исключением ненулевых $\epsilon_i$, появляющихся в сумме.

\subparagraph{b.} \textit{// here goes some text and code}

\setcounter{equation}{6}

\begin{equation}
{\sum}_{E'} = {\sum}_E + (-1)^{|E'|} \prod_{1 \leqslant i \leqslant n} S_i(E').
\end{equation}

\noindent
Вклад наименьшего элемента 0̸  в эту сумму — нулевой; значит, начнем с его последователя, который может быть \{1\}, \{$n$\} или какое-нибудь одноэлементное множество в соответствии с порядком, заданным на [1, $n$]. Можно заметить, что $S_i(E') = S_i(E) \pm a_{ij}$, где $\{j\} = E \triangle E'$.

\newpage

\lhead{146}
\rhead{\small\textit{$I$ \quad Алгоритмика и программирование на языке Ада}}

\noindent
В этой записи, как в алгоритме 12, $\pm$ должен пониматься как $+$, если $E \subset E'$ и $-$, если $E' \subset E$.

Чтобы оценить ${\sum}_{E'}$, исходя из ${\sum}_E$ с использованием 
соотношения (7), нужно осуществить $n$ сложений (для подсчета каждого $S_i (E')$), затем $n - 1$ перемножений: $S_1(E') \times \cdots \times S_n(E')$, и, наконец, 1 сложение. Имеем $2^n - 2$ операций для осуществления (7), при этом первый член $\sum = -{\prod}_{1 \leqslant i \leqslant n} \alpha_{in}$ требует $n - 1$ перемножений; это доказывает сформулированный результат о сложности. Сложность \textit{O}($n2^n$) значительна, но остается того же порядка, что и сложность, индуцированная определением ($n!(n - 1)$ перемножений и $n! - 1$ сложений). Формула Стирлинга позволяет сравнить эти два значения сложности: $ n \cdot n!/(n \cdot 2^n) \approx (n/2e)^n \sqrt{2\pi n}$.

\paragraph{22. Перманент матрицы (продолжение)}

\subparagraph{a.} Правая часть может рассматриваться как многочлен (от переменных $a_{ij}$), равный ${\sum}_\alpha \mu_\alpha a_{1 \alpha(1)} ... a_{n \alpha(n)}$, где сумма распространяется на все отображения [1, $n$] в [1, $n$]. Коэффициент $\mu_\alpha$ задан формулой

\begin{equation*}
\mu_\alpha = \sum_\omega \mu_\alpha(\omega) \quad \text{с} \quad \mu_\alpha(\omega) = \omega_1 \text{ ... } \omega_{n - 1} \omega_{\alpha(1)} \text{ ... } \omega_{\alpha(n)},
\end{equation*}

\noindent
в которой полагаем $\omega_n = 1$. Если $\alpha$ — перестановка, то каждое $\mu_\alpha(\omega)$, присутствующее в сумме $\mu_\alpha$, равно 1 и, следовательно, $\mu_\alpha = 2^{n - 1}$. Напротив, если $\alpha$ не является перестановкой, то сумма $\mu_\alpha$ — нулевая. Действительно, образ $\alpha$ отличен от [1, $n$], и различаем два случая: \newline
\indent ($i$) $\exists$ $k < n$, не принадлежащий образу $\alpha$, \newline
\indent ($ii$) $\exists$ $k < n$, дважды полученный из $\alpha$.

В обоих случаях члены $\mu_\alpha(\omega)$, присутствующие в сумме, группируются попарно, один соответствуя $\omega_k = 1$, другой — $\omega_k = -1$, и взаимно уничтожаются (в случае ($ii$) $\mu_\alpha(\omega) = \omega_k$).

\subparagraph{b.} Формула пункта \textbf{a} может быть записана в следующем виде:

\begin{equation*}
\frac{per A}{2} = \sum \omega_1 \text{ ... } \omega_{n - 1} \prod_{1 \leqslant i \leqslant n} (a_{in} + \omega_1 a_{i1} + \cdots + \omega_{n - 1} a_{in - 1})/2.
\end{equation*}

\noindent
Как и в предыдущем упражнении, вычисление перманента получается генерированием перебора при линейной упорядоченности на $\{-1, 1\}^{n - 1}$, в которой два последовательных элемента отличаются только одной компонентой. Если для $\omega \in \{-1, 1\}^{n - 1}$ и $i \leqslant n$ положить

\begin{equation*}
S_i(\omega) = (a_{in} + \omega_1 a_{i1} + \cdots + \omega_{n - 1} a_{in - 1})/2,
\end{equation*}

\newpage

\lhead{\small\textit{Решения упражнений}}
\rhead{147}

\noindent
то можно, благодаря перебору на $\{-1, 1\}^{n - 1}$, вычислить последовательно ${\sum}_\omega = {\sum}_{\rho \leqslant \omega}$... , используя формулу

\begin{equation}
	\begin{split}
	{\sum}_{\omega'} \quad = \quad &{\sum}_\omega + {\omega'}_1 \text{ ... } {\omega'}_{n - 1} \prod_{1 \leqslant i \leqslant n} S_i(\omega'),
	\\
	&\text{где } \omega' \text{ — последователь для } \omega \text{ в } \{-1, 1\}^{n - 1}.
	\end{split}
\end{equation}

\noindent
Если $j$ является индексом, по которому различаются два слова $\omega$ и $\omega'$, то сумма $S_i(\omega')$ вычисляется, исходя из $S_i(\omega)$, через $S_i(\omega') = S_i(\omega) - a_{ij}$, если $\omega_j = 1$, и через $S_i(\omega') = S_i(\omega) + a_{ij}$, если $\omega_j = -1$. \newline

\textit{// here goes some code}

\subparagraph{Алгоритм 13.} Вычисление перманента в кольце, где 2 обратимо \newline

С практической точки зрения для генерации адекватного перебора \linebreak $\{-1, 1\}^{n - 1}$, выбираем соответствие между $\{0, 1\}$ и $\{-1, 1\}$ вида $\epsilon \mapsto (-1)^\epsilon$ и классическую генерацию кода Грея на $\{0, 1\}^{n - 1}$, что достигается с помощью алгоритма 13. Мультипликативная сложность получается, если заметить, что нужно вычислить $2^n - 1$ членов ${\sum}_\omega$ , каждый из которых требует $n - 1$ перемножений (см. формулу (8)), значит, всего $2^{n - 1}(n - 1)$ произведений, к которым нужно добавить последнее умножение на 2. С точки зрения сложений первоначальный член ${\sum}_{(1,...,1)}$ требует $n(n - 1)$ сложений и $n$ делений на 2, тогда как общий член ${\sum}_\omega$ вычисляется, исходя из предыдущего, с помощью $n$ сложений; наконец, нужно сложить все эти члены, что требует в целом $n(n - 1) + (2^{n - 1} - 1)(n + 1)$ сложений и $n$ делений на 2. Заметим относительно предыдущего упражнения, что сложность была приблизительно разделена на 2.

\newpage

\lhead{148}
\rhead{\small\textit{$I$ \quad Алгоритмика и программирование на языке Ада}}

\subparagraph{c.} Рассмотрения полностью аналогичны предыдущему пункту, если
только невозможно деление на 2; деление (точное) на $2^{n - 1}$ будет иметь
место уже в конце. Результатом является алгоритм 14. \newline

\textit{// here goes some code}

\paragraph{23. Массив инверсий подстановки}

\subparagraph{d.} Массив инверсий перестановки $\alpha$ имеет вид (0, 0, 0, 1, 4, 2, 1, 5, 7). Свойство $0 \leqslant \alpha_k < k$ легко получается из того, что имеется точно $k - 1$ целых чисел, заключенных строго между 0 и $k$. Массив инверсий возрастающей перестановки интервала [1, $n$] есть, очевидно, (0, 0, ... , 0), и таблица для единственной убывающей перестановки —  (0, 1, 2, ... , n).

\subparagraph{e.} Можно использовать тот факт, что $a_{\alpha(j)}$ есть число индексов таких, что $i > j$ и $\alpha(i) < \alpha(j)$, что приводит к нижеследующему алгоритму: \newline

\textit{// here goes some code} \newline

Сложность полученного способа, конечно, имеет порядок квадрата длины перестановки.

\newpage

\lhead{\small\textit{Решения упражнений}}
\rhead{149}

\subparagraph{f.} Пусть $a$ — элемент из $[0, 1[ \times [0, 2[ \times \cdots \times [0, n[$. Построим перестановку $\alpha$, для которой $a$ является массивом инверсий, следующим способом: \newline

$\bullet$ элемент $n$ помещаем в массив, индексированный с помощью [1, $n$], представляющий $\alpha$, оставляя $a_n$ \textbf{пустых ячеек} справа от $n$; это означает в точности, что $\alpha^{-1}(n) = n - a_n + 1$;

$\bullet$ затем помещаем $n - 1$ в массив $\alpha$, оставляя $\alpha_{n - 1}$ \textbf{пустых ячеек} справа от $n - 1$;

$\bullet$ продолжаем, зная, что на $k$-м этапе этого процесса $k - 1$ величин уже размещены в массиве, следовательно, в массиве $\alpha$ остается $n - k$ свободных мест, и, с другой стороны, величина $\alpha_{n - k}$ строго меньше, чем $n - k$. \newline

\textit{// here goes some code} \newline

В алгоритме 15 использована оптимизация: свободные места в массиве, представляющем перестановку $\alpha$, отмечены числами 1, что позволяет не помещать это последнее значение в массив, представляющий $\alpha$, в конце алгоритма.

\paragraph{24. Перебор перестановок транспозициями $(i, i + 1)$}

\subparagraph{a.} Рассуждаем индукцией по $n$, при этом случаи $n = 1$ и $n = 2$ очевидны. С помощью перестановки $\sigma$ интервала [1, $n$] можно построить $n + 1$ перестановок $\sigma^1, \sigma^2, \text{ ... } , \sigma^{n + 1}$ интервала [1, $n + 1$], где перестановка $\sigma^i$ получается включением в $\sigma$ элемента $n + 1$ на $i$-ое место; например,

\newpage

\lhead{150}
\rhead{\small\textit{$I$ \quad Алгоритмика и программирование на языке Ада}}

\noindent
если $\sigma = (5 2 4 1 3)$, то

\begin{equation*}
	\begin{split}
	\sigma^1 &= \text{(\underline{6} 5 2 4 1 3)}, \quad \sigma^2 = \text{(5 \underline{6} 2 4 1 3), ... ,}
	\\
	&\sigma^5 = \text{(5 2 4 1 \underline{6} 3)}, \quad \sigma^6 = \text{(5 2 4 1 3 \underline{6})}.
	\end{split}
\end{equation*}

\noindent
Если $\sigma_1, \sigma_2, \text{ ... } , \sigma_{n!}$ и есть такая последовательность перестановок на [1, $n$], то соответствующую последовательность перестановок интервала [1, $n + 1$] получаем следующим образом:

\begin{equation*}
	\begin{split}
	&\sigma_1^1, \sigma_1^2, \text{ ... } , \sigma_1^{n + 1}, \quad \sigma_2^{n + 1}, \sigma_2^n, \text{ ... } , \sigma_2^1
	\\
	\sigma_3^1, &\sigma_3^2, \text{ ... } , \sigma_3^{n + 1}, \quad \sigma_4^{n + 1}, \sigma_4^n, \text{ ... }, \sigma_4^1 \quad \text{и т.д.}
	\end{split}
\end{equation*}

\subparagraph{b.} Действуем индукцией по $n$; знаем, что \textit{b} —  последующий элемент для $a$ в знакопеременном лексикографическом порядке —  получается изменением \textit{одной} компоненты $а$. Предположим сначала, что эта компонента — последняя; тогда имеем $b_n = a_n \pm 1$ и $b_j = a_j$ для $1 \leqslant j \leqslant n - 1$. Если $i$ — индекс $n$ в $\alpha$ (т.е. $\alpha(i) = n$), то имеем $\beta = \alpha \circ (i, i - 1)$ в случае $b_n = a_n + 1$, и $\beta = \alpha \circ (i, i + 1)$ в случае $b_n = a_n - 1$, при этом запись $(j, k)$ означает транспозицию индексов j и k.

Теперь предположим, что компонента, по которой различаются $a$ и $b$, не является последней. Поскольку $b$ — последующий элемент для $a$ в знакопеременном лексикографическом порядке, имеем $a_n = b_n = 0$ или $a_n = b_n = n$ и $b_{[1..n - 1]}$ есть последующий элемент для $a_{[1..n - 1]}$ в лексикографическом знакопеременном произведении $[0, 1[ \times [0, 2[ \times \cdots \times [0, n - 1[$, причем компонентой с самым большим индексом массива инверсий является та, которая меняется быстрее всех во время перебора в лексикографическом знакопеременном порядке. Если $\alpha'$ (соответственно, $\beta'$) означает перестановку [1, $n - 1$], для которой $a_{[1..n - 1]}$ (соответственно, $b_{[1..n - 1]}$) массив инверсий, то $\beta'$ получается из $\alpha'$ транспозицией двух последовательных элементов (гипотеза индукции). Тогда утверждение верно также и для $\alpha$ и $\beta$, поскольку элемент $n$ находится в этих перестановках либо на месте $n$ (случай $a_n = b_n = 0$), либо на месте 1 (случай $a_n = b_n = n$).

\subparagraph{c.} Пусть $\alpha$ — перестановка интервала [1, $n$] и $a = (a_1, a_2, \text{ ... }, a_n)$ — ее массив инверсий. По предыдущему сигнатура перестановки $\alpha$ является также сигнатурой $a$ в знакопеременном лексикографическом произведении. Алгоритм вычисления последующего элемента в знакопеременном лексикографическом произведении приводит к алгоритму 16-А.

В начале тела основного цикла этого алгоритма делается попытка опустить элемент $q$ (применить транспозицию ($\alpha^{-1}(q) - 1, a^{-1}(q)$) к перестановке $\alpha$) или же поднять его.

\newpage

\lhead{\small\textit{Решения упражнений}}
\rhead{151}

Фактически, бесполезно приниматься за предварительное вычисление массива инверсий $a$. Достаточно вычислить при необходимости элемент $a_q$, что может быть реализовано одновременно с вычислением $\alpha^{-1}(q)$ благодаря алгоритму 16-В. В этом втором алгоритме результирующим значением $i$ является $\alpha^{-1}(q)$. \newline

\textit{// here goes some code}

\paragraph{25. Принцип включения-исключения или формула решета}

\subparagraph{a.} Первая формула удобно получается индукцией по $\abs{I}$, числу элементов $I$, с использованием хорошо известной формулы $\abs{A \cup B} = \abs{A} + \abs{B} - \abs{A \cap B}$. Вторая получается переходом к дополнениям. Чтобы получить формулу Сильвестра, достаточно записать:

\begin{equation*}
\bigcap_{i \in I} \overline{X_i} = \overline{\bigcup_{i \in I} X_i} = X - \bigcup_{i \in I} X_i,
\end{equation*}

\noindent
затем применить первую формулу.

\subparagraph{b.} Пусть X — множество всех перестановок на [1, $n$] и $X_i$ —  множество перестановок, имеющих $i$ фиксированной точкой, $1 \leqslant i \leqslant n$. Искомое число $\sigma_n$:

\begin{equation*}
\sigma_n = \abs[\Big]{\bigcup_{i \in [1, n]} \overline{X_i}} = \sum_{J \subset [1, n]} (-1)^{\abs{J}} \abs[\Big]{\bigcap_{i \in J} X_i}.
\end{equation*}

\noindent
Множество $\bigcap X_i$ есть множество $J$ перестановок на [1, $n$] и, следовательно, содержит $(n - \abs{J})!$ элементов, откуда:

\begin{equation*}
\sigma_n = \sum_{J \subset [1, n]} (-1)^{\abs{J}} (n - \abs{J})! = \sum_{k = 0}^n (-1)^k C_n^k(n - k)! = \sum_{k = 0}^n (-1)^k \frac{n!}{k!}
\end{equation*}

\newpage

\lhead{152}
\rhead{\small\textit{$I$ \quad Алгоритмика и программирование на языке Ада}}

\subparagraph{c.} Положим здесь $X$ равным интервалу [1, $n$] и для $1 \leqslant i \leqslant k$ пусть $X_i$ — множество элементов из $X$, которые кратны $p_i$. Тогда имеем:

\begin{equation*}
\phi(n) = \abs[\Big]{\bigcap_{i \in [1, k]} \overline{X_i}} = \sum_{J \subset [1, k]} (-1)^{\abs{J}} \abs[\Big]{\bigcap_{i \in J} X_i}
\end{equation*}

\noindent
Множество $\bigcap X_i$ здесь является множеством элементов из X, кратных ${\prod}_{i \in J} p_i$; но если $d$ является делителем $n$, имеется точно $n / d$ элементов из $X$, кратных $d$, откуда:

\begin{equation*}
\phi(n) = n \sum_{J \subset [1, k]} \frac{(-1)^{\abs{J}}}{{\prod}_{i \in J} p_i} = n (1 - \frac{1}{p_1}) (1 - \frac{1}{p_2}) \text{ ... } (1 - \frac{1}{p_k}).
\end{equation*}

\subparagraph{d.} Обозначим через $X$ множество всех отображений из [1, $n$] в [1, $n$] и для $1 \leqslant i \leqslant n$ через $X_i$ — множество отображений из X, не имеющих $i$ в их образе; выберем в качестве весовой функции на $X$ функцию $p(\alpha) = a_{1\alpha(1)} a_{2\alpha(2)} \text{ ... } a_{n\alpha(n)}$: тогда нужно вычислить вес множества $S_n$ всех перестановок на [1, $n$]. Заметим, что $S_n = {\bigcap}_{i \in [1, n]} \overline{X_i}$ и, значит, $per A =$ \linebreak ${\sum}_{J \subset [1, n]} (-1)^{\abs{J}} p({\bigcap}_{i \in J} X_i)$. Но ${\bigcap}_{i \in J} X_i$ есть множество всех отображений, образы которых не встречаются в $J$, следовательно:

\begin{equation*}
p(\bigcap_{i \in J} X_i) = \sum_{\alpha : [1, n] \rightarrow \overline{J}} a_{1\alpha(1)} \text{ ... } a_{n\alpha(n)} = \sum_{j \not\in J} a_{1j} \sum_{j \not\in J} a_{2j} \text{ ... } \sum_{j \not\in J} a_{nj}
\end{equation*}

\noindent
отсюда получаем формулу $per A = {\sum}_{J \subset [1, n]} (-1)^{\abs{J}} {\prod}_{i = 1}^n {\sum}_{j \not\in J} a_{ij}$, которая при замене $J$ его дополнением дает формулу Райзера.

\paragraph{26. Произведение многочленов, заданных массивами}

\subparagraph{a.} Алгоритм справа дает функцию умножения двух многочленов $Р$ и $Q$, где многочлен $R$ степени deg $Р$ + deg $Q$ (который дает результат в конце алгоритма) должен быть предварительно инициализирован нулем.

\subparagraph{b.} Изучая предыдущий алгоритм, устанавливаем, что его сложность, как по числу перемножений, так и сложений, равна произведению высот двух многочленов: (deg $P$ + 1) $\times$ (deg $Q$ + 1) — обычно высо-

\newpage

\lhead{\small\textit{Решения упражнений}}
\rhead{153}

\noindent
той многочлена называют число его ненулевых коэффициентов, но в этом алгоритме, который не учитывает случай нулевых коэффициентов, можно рассматривать высоту многочлена как число всех коэффициентов. Значит, возможно улучшить предыдущий алгоритм, исключив все ненужные перемножения: это сделано в алгоритме 17. В противовес тому, что можно было бы подумать, эта оптимизация вовсе не смехотворная и активно применяется при умножении разреженных многочленов.

\paragraph{27. Возведение в степень многочленов, заданных массивами}

\subparagraph{a.} Очень просто вычислить сложность алгоритма возведения в степень последовательными умножениями, если заметить, что когда $P$ — многочлен степени $d$, то $P^i$ — многочлен степени $id$. Если обозначить $C_{mul}(n)$ сложность вычисления $P^n$, то рекуррентное соотношение $C_{mul}(i + 1) =$ \linebreak $C_{mul}(i) + (d + 1) \times (id + 1)$ дает нам:

\begin{equation*}
C_{mul}(n) = (d + 1) \sum_{i = 1}^{n - 1} id + 1 = \frac{n^2 d(d + 1)}{2} + \frac{n(d + 1)(2 - d)}{2} - (d + 1).
\end{equation*}

\subparagraph{b.} Что касается возведения в степень с помощью дихотомии (т.е. повторяющимся возведением в квадрат), вычисления несколько сложнее: зная $P^{2^i}$, вычисляем $P^{2^{i + 1}}$ с мультипликативной сложностью $(2^i d + 1)^2$. Как следствие имеем:

\begin{equation*}
	\begin{split}
	C_{sqr}(2^l) = \sum_{i = 0}^{l - 1} (2^i d + 1)^2 &= \frac{d^2 (4^l - 1)}{3} + 2d(2^l - 1) + l =
	\\
	&= \frac{d^2 n^2}{3} + 2nd + \log_2 n - 2d - \frac{d^2}{3}
	\end{split}
\end{equation*}

\noindent
Предварительное заключение, которое можно вывести из предыдущих вычислений, складывается в пользу дихотомического возведения в степень: если $n$ есть степень двойки (гипотеза ad hoc), этот алгоритм еще выдерживает конкуренцию, даже если эта победа гораздо скромнее в данном \linebreak

\newpage

\lhead{154}
\rhead{\small\textit{$I$ \quad Алгоритмика и программирование на языке Ада}}

\noindent
контексте ($n^2 d^2 / 3$ против $n^2 d^2 / 2$), чем когда работаем в $\mathbb{Z} / p \mathbb{Z}$ ($2 \log_2 n$ против $n$).

Но мы не учли корректирующие перемножения, которые должны быть выполнены, когда показатель не является степенью 2-х. Если $2^{l + 1} - 1$, нужно добавить к последовательным возведениям в квадрат перемножения всех полученных многочленов. Умножение многочлена $P^{(2^i - 1)d}$ степени $(2^i - 1)d$ на многочлен $P^{2^i d}$ степени $2^i d$ вносит свой вклад из $((2^i - 1)d + 1) \times (2^i d + 1)$ умножений, которые, будучи собранными по всем корректирующим вычислениям, дают дополнительную сложность:

\begin{equation*}
	\begin{split}
	C_{sqr}'(2^{l + 1} - 1) &= \sum_{i = 1}^l ((2^i - 1)d + 1) \times (2^i d + 1) =
	\\
	&= \frac{d^2 n^2}{3} - \frac{4d^2 n}{3} + 2nd -2d^2 + d(1 - \floor{\log_2 n} ).
	\end{split}
\end{equation*}

\noindent
Теперь можно заключить, что дихотомическое возведение в степень не всегда является лучшим способом для вычисления степени многочлена с помощью перемножений многочленов. Число перемножений базисного кольца, которые необходимы, $C_{sqr}(n)$ — в действительности заключено между $C_{sqr}(2^{\floor{log_2 n}})$ и $C_{sqr}(2^{\floor{log_2 n}}) + C'(2^{\floor{log_2 n} + 1} - 1)$, т.е. между $n^2 d^2 / 2$ и $2n^2 d^2 / 3$, тогда как простой алгоритм требует всегда $n^2 d^2 / 2$ перемножений. В частности, если исходный многочлен имеет степень, большую или равную 4, возведение в степень наивным методом  требует меньше перемножений в базисном кольце, чем бинарное возведение в степень, когда $n$ имеет форму $2^l - 1$. \newline


Можно пойти еще дальше без лишних вычислений: можно довольно просто доказать, что если $n$ имеет вид $2^l + 2^{l - 1} + c$ (выражения, представляющие двоичное разложение $n$), то метод вычисления последовательными перемножениями лучше метода, использующего возведение в квадрат (этот последний метод требует корректирующего счета ценой, по крайней мере, $n^2 d^2 / 9$). Все это доказывает, что наивный способ является лучшим для этого класса алгоритмов, по крайней мере в половине случаев. \newline

Действительно, МакКарти [124] доказал, что дихотомический алгоритм возведения в степень оптимален среди алгоритмов, оперирующих повторными умножениями, если действуют с плотными многочленами (антоним к разреженным) по модулю $m$, или с целыми и при условии оптимизации возведения в квадрат для сокращения его сложности наполовину (в этом

\newpage

% случае сложность ...

\end{document} 
