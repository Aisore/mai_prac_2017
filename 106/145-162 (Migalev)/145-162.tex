\documentclass{mai_book}

\defaultfontfeatures{Mapping=tex-text}
\setmainfont{DejaVuSans}
\setdefaultlanguage{russian}

\begin{document}

\lhead{\small\textit{Решения упражнений}}
\rhead{145}

\noindent
[1, \textit{n}] в [1, \textit{n}], при этом коэффициент $\mu_\alpha$ определяется суммой

\begin{equation*}
\mu_{\alpha} = \sum_{\epsilon_i\in\{0,1\}} (-1)^{\epsilon_1+...+\epsilon_n} \epsilon_{\alpha(1)}...\epsilon_{\alpha(n)}.
\end{equation*}

\noindent
Если $\alpha$ — перестановка, коэффициент $\mu_\alpha$ равен $(—1)^n$, поскольку единственный ненулевой член суммы — это тот, для которого все $\epsilon_i$, равны 1. В противном случае можно разложить $\mu_\alpha$ следующим образом:

\begin{equation*}
\sum_{\epsilon_i, i \neq k} (-1)^{\epsilon_1+...+\epsilon_n} \sum_{\epsilon_k=0,1} (-1)^{\epsilon_k} \epsilon_{\alpha(1)}...\epsilon_{\alpha(n)},
\end{equation*}

\noindent
где \textit{k} — элемент из [1, \textit{n}], который не принадлежит образу $\alpha$, и хорошо видно, что внутренний член (сумма по $\epsilon_k$) равен нулю; это доказывает, что $\mu_\alpha$ в этом случае равно нулю. Второе искомое выражение (суммы которого записываются на множествах) просто выводится из первого, исключением ненулевых $\epsilon_i$, появляющихся в сумме.

\setcounter{equation}{6}

\begin{equation}
{\sum}_{E'} = {\sum}_E + (-1)^{|E'|} \prod_{1 \leq i \leq n} S_i (E').
\end{equation}

\noindent
Вклад наименьшего элемента 0̸  в эту сумму — нулевой; значит, начнем с его последователя, который может быть {1}, {\textit{n}} или какое-нибудь одноэлементное множество в соответствии с порядком, заданным на [1,\textit{n}]. Можно заметить, что $S_i(E') = S_i(E) \pm a_{ij}$, где $\{j\} = E \triangle E'$.

\end{document} 
