\documentclass{mai_book}

\defaultfontfeatures{Mapping=tex-text}
\setmainfont{DejaVuSerif}
\setdefaultlanguage{russian}

%\clearpage
\setcounter{page}{270} % ВОТ ТУТ ЗАДАТЬ СТРАНИЦУ
\setcounter{thesection}{5} % ТАК ЗАДАВАТЬ ГЛАВЫ, ПАРАГРАФЫ И ПРОЧЕЕ.
% Эти счетчики достаточно задать один раз, обновляются дальше сами
% \newtop{ЗАГОЛОВОК}  юзать чтобы вручную поменть заголовок вверху страници

\begin{document}


\begin{center} %Решения упражнений
\ \newline\ \newline
\Large \textbf{Решения упражнений}\\
\ \newline
\end{center}
\noindent\textbf{1. Десятичные цифры простых чисел}\\
\ \newline
\hspace*{15pt}Пусть $l$ -- длина $B$, т.е. $10^{l-1}\leqslant B <10^l$\: и \:$k\geqslant 0$. Тогда числа вида\linebreak
$n\cdot10^{k+1} + B\cdot10^k + c$ являются числами десятичная запись которых\linebreak
содержит последовательность $В$, если $ 0 \leqslant c < 10^k.$ Если $k > 0$, то\linebreak
можно взять $с$ взаимно простым с 10 и тогда числа $10^{k+1}$ и $B\cdot10^k + c$\linebreak
будут взаимно просты. Поэтому можно применить теорему Дирихле. В\linebreak действительности оказывается, что существует бесконечно много про­-\linebreak
стых чисел, содержащих фиксированную последовательность цифр $B$ в\linebreak
заранее заданных позициях.\\
\ \newline
\noindent\textbf{2. Вычисление НОД}\\

\hspace*{0pt} Допустим, что $m>n$. Тогда $a^m - b^m = (a^n-b^n)a^{m-n}+(a^{m-n}-b^{m-n}b^n)$.\linebreak 
Это тождество и условия взаимной простоты $a$ и $b$ дает\newline
$$\text{НОД}(a^m-b^m,\ a^n-b^n)=\text{НОД}(a^{m-n}-b^{m-n},\ a^n-b^n)=$$
	$$=\text{НОД}(a^{m\ mod\ n}-b^{m\ mod\ n},\ a^n-b^n),$$
Повторяя это соотношение как в алгоритме Евклида, получим искомый результат.
\ \newline

\noindent\textbf{3. Алгоритм Евклида и непрерывные дроби}\\
\\
\hspace*{10pt} \textbf{a.}\ Пусть $a/b$ — рациональное несократимое число (с $b > 0$). Тогда\linebreak 
последовательность частных, полученных в алгоритме Евклида, при­-\linebreak
мененном к $а$ и $b$, дает разложение $а/b$ в непрерывную дробь. К тому\linebreak
же все частные, за исключением, быть может, первого, положительны,\linebreak
если используемое деление является обычным делением Евклида.\linebreak

 \textbf{b.}\ Рассмотрим непрерывные дроби $s_i = [c_i,c_{i+1},..,c_m]$ и\linebreak
$t_i=[d_i,d_{i+1},...,d_n]$. Тогда, очевидно, имеем $s_i=c_i+1/s_{i+1}$ и \linebreak
аналогичное соотношение для $t$ и $d$. Отсюда получаем $s_i>1$ для $i<m$,\linebreak
и, следовательно, $[s_i] = c_i$. кроме того, по предположению, $s_0=$\linebreak
$=[c_0,c_1,...,c_m]=[d_0,d_1,...,d_n]=t_0$. Используя предыдущее соотношение,\linebreak
получаем $c_0=d_0$ (целые части $s_0$ и $t_0$) и $s_1=t_1$. Затем постепенно\linebreak \pagebreak

%===============================================================
% 271 страница

\newtop{Решения упражнений} %!!!!!!!!!!!!!!!!!!!!

\noindent доказываем, что $c_i=d_i$. Этот процесс заканчивается на наименьшем \linebreak
из чисел $m$ и $n$. Допустим, что это $m$. Тогда $s_m=t_m$ и $c_m=d_m$ через\linebreak
предшествующую рекуррентность. Кроме того, $s_m=c_m$ по определе­нию $s$.\linebreak
В результате\:\: $c_m=d_m$\:\: и\:\: $m=n$.\newline
\ \newline
\hspace*{15pt}\textbf{c.} Достаточно рассмотреть последние деления алгоритма Евклида:\nolinebreak $$r_{n-2}=r_{n-1}a_{n-1}+1\quad\text{ и }\quad r_{n-1}=1\times r_{n-1} \quad\text{ с\: } r_{n-1}>1.$$
Последнее деление можно заменить на следующее:\:\: $r_{n-1}=1\times (r_{n-1}-1)+1$ \linebreak
без появления нулевого частного и закончить деление на $1=1\times1$.\linebreak
Это означает, что $$[c_0,c_1,c_2,...,c_n]=[c_0,c_1,c_2,...,c_n-1,1],\quad\text{если\: } c_n > 1.$$

Это эквивалентно факту, что любое целое число $a$ допускает ровно два
разложения в непрерывную дробь: $[a]$ и $[a-1,1]$.\newline\\
\hspace*{15pt}\textbf{d.} Вот часть ответа (развиваемая потом в упражнении 6):
\begin{gather*}
	[a_0]=a_0,\quad [a_0,a_1]=\frac{a_0a_1+1}{a_1},\quad
	[a_0,a_1,a_2]=\frac{a_0a_1a_2+a_0a_1+a_1a_2}{a_1a_2+1}...\text{\:,}\\
	F_4/F_3=[1,1,2]\quad \text{и}\quad F_5/F_4=[1,1,1,2].
\end{gather*}
\\
\noindent\textbf{4. Многочлены континуаты}\\
\\
\hspace*{15pt}\textbf{a.} Легко видеть, что $F_n=K_n(1,...,1)$. Это означает, что в $K_n$\linebreak
имеется $F_n$ слагаемых.\\
\\
\hspace*{15pt}\textbf{b.} Нетрудно проверить, что свойство верно для $n=$ —1, 0, 1.\linebreak
Рассмотрим моном $X_1...X_n$ . Можно разделить исключения примыкающих\linebreak
пар на две категории: те, которые исключают пару $X_nX_{n-1}$, и те,\linebreak
которые оставляют $X_n$. По предположению индукции, первая категория\linebreak
дает все одночлены от $K_{n-2}$, а вторая — все одночлены от $K_{n-1}$,\linebreak
умноженные на $X_n$. Отсюда результат (обнаруженный впервые Эйлером).\linebreak
\\
\hspace*{15pt}\textbf{c.} Следовательно, континуанты обладают зеркальной симметрией:\linebreak
$K_n(X_1,...,X_n)=K_n(X_n,...,X_1)$. Можно определить последователь­-\linebreak
ность $K_n$ следующим образом:
$$K_n=X_1K_{n-1}(X_2,...,X_n)+K_{n-2}(X_3,...,X_n).$$
Соотношение, которое будет использоваться в следующих упражне­ниях.\newpage

%========================================================================
%272
Требуемое тождество доказывается легко. Оно указывает, что в\linebreak
по­ле рациональных дробей
$$ \frac{K_n(X_1,...,X_n)}{K_{n-1}(X_2,...,X_n)}=[X_1,X_2,...,X_n]=X_1+\frac{1}{X_2+...}.$$
Если $r_{i-1}=r_ia_i+r_{i+1}$, где $r_n=1$ и $r_{n+1}=0$ есть последовательность\linebreak
делений алгоритма Евклида, примененного к целым взаимно простым\linebreak
числам, то $r_i=K_{n-i}(a_{i+1},...,a_n)$.\newline
\\
\noindent\textbf{5. Континуаты (продолжение)}\\
\\
\hspace*{15pt}\textbf{a.} Нетрудно доказать, используя индукцию, что рассматриваемое\linebreak
произведение равно матрице
$$\begin{pmatrix}
	K_n(X_1,...,X_n)& K_{n-	1}(X_1,...,X_{n-1})\\
	K_{n-1}(X_2,...,X_n)& K_{n-2}(X_2,...,X_{n-1})
\end{pmatrix}$$
\noindent Этот результат дает другое доказательство зеркальной симметричности\linebreak
многочленов континуант.\newline
\\
\hspace*{15pt}\textbf{b.} Вычислив определитель, находим искомое равенство (с точно­стью\linebreak
до индексов). Следовательно, значения $K_n(a_1,...,a_n)$ и\linebreak
$K_{n-1}(a_2,...,a_n)$ дают взаимно простые целые числа.\\
\\
\hspace*{15pt}\textbf{c.} Простая индукция показывает, что искомое произведение равно
$$\begin{pmatrix}
	K_{n+2}(X_1,...,X_{n+2})& K_{n+1}(X_2,...,X_{n+2})\\
	K_n(X_1,...,X_n)& K_{n-1}(X_2,...,X_n)
\end{pmatrix}$$
\noindent Определитель этой матрицы (что и требовалось доказать) равен
\begin{eqnarray*}
	X_{n+2}\Big( K_{n-1}(X_2,...,X_n) K_{n+1}(X_1,...,X_{n+1})-\quad\\
	-K_n(X_1,...,X_n)K_n(X_2,...,X_{n+1}) \Big),
\end{eqnarray*}
т.е.\:\: $(-1)^{n+1}X_{n+2}$\:\: по предыдущему пункту\\
\\
\noindent\textbf{6. Разложение в непрерывную дробь}\\
\\
\hspace*{15pt}\textbf{a.} Это решение уже намечено в упражнении 3:
$$[a_0]=a_0,\quad [a_0,a_1]=\frac{a_0a_1+1}{a_1},\quad [a_0,a_1,a_2]=\frac{a_0a_1a_2+a_0a_1+a_1a_2}{a_1a_2+1}...$$
\pagebreak

%===================================================================
%273
\noindent Формула выглядит так:
$$[a_0,a_1,...,a_n]=\frac{K_{n+1}(a_0,a_1,...,a_n)}{K_n(a_1,...,a_n)},$$
и доказывается по индукции. Попутно использовалось соотношение\linebreak
$[a_0,...,a_n]=[a_0,[a_1,...,a_n]]$ и равенство, выражающее зеркальность\linebreak
многочленов континуант.\\
\\
\hspace*{15pt}\textbf{b.}





\end{document}
